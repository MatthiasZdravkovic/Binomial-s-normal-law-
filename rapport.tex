\documentclass{article}
\usepackage{graphicx} % Required for inserting images
\usepackage{amsmath}
\usepackage{marginnote}
\usepackage{amssymb}

\title{Projet Maths}
\author{Matthias Zdravkovic, Ameen Mohd Fairuz}
\date{Mai 2023}

\begin{document}


\maketitle

\section{Développements mathématiques}

\reversemarginpar\marginnote{1. a)}[0cm]

On part de la loi normale bidimensionnelle :
$$f_Z(z) = \frac{1}{2\pi \sqrt{\det \Sigma}}exp(-\frac{1}{2}(z-\mu)^T\Sigma^{-1}(z-\mu))$$
On pose
$$\Sigma = U \cdot \Lambda \cdot U^T$$
$$ \Leftrightarrow \det \Sigma = \det U \cdot \det \Lambda \cdot \det (U^T) $$
\begin{center}
$(\det U)^{2}=1$ car c'est une matrice orthogonale
\end{center}
$$\Leftrightarrow \det \Sigma =\det \Lambda$$
avec $\Sigma \in M_2(\mathbb{R})$ matrice symétrique positive définie, $U \in M_2(\mathbb{R})$ matrice orthogonale et $\Lambda \in M_2(\mathbb{R})$ matrice diagonale à coefficients positifs.\\
On pose
\[
U = \begin{pmatrix}
    \cos(\theta) & -\sin(\theta) \\
    \sin(\theta) & \cos(\theta)
\end{pmatrix}
\]
et 

\[
\Lambda = \begin{pmatrix}
    a & 0 \\
    0 & b
\end{pmatrix}
\]
avec 
$$ \Sigma^{-1} = U \cdot \Lambda^{-1} \cdot U^{-1}$$
$$\Leftrightarrow \Sigma^{-1} = \begin{pmatrix} \cos(\theta) & -\sin(\theta) \\ \sin(\theta) & \cos(\theta) \end{pmatrix} \cdot \begin{pmatrix} \frac{1}{a} & 0 \\ 0 & \frac{1}{b} \end{pmatrix} \cdot \begin{pmatrix} \cos(\theta) & -\sin(\theta) \\ \sin(\theta) & \cos(\theta) \end{pmatrix}^{-1}$$
$$\Leftrightarrow \Sigma^{-1} = \begin{pmatrix} \cos(\theta) & -\sin(\theta) \\ \sin(\theta) & \cos(\theta) \end{pmatrix} \cdot \begin{pmatrix} \frac{1}{a} & 0 \\ 0 & \frac{1}{b} \end{pmatrix} \cdot \frac{1}{\cos^{2} \theta + \sin^{2} \theta}\begin{pmatrix} \cos(\theta) & \sin(\theta) \\ -\sin(\theta) & \cos(\theta) \end{pmatrix}$$
$$\Leftrightarrow \Sigma^{-1} = \begin{pmatrix} \cos(\theta) & -\sin(\theta) \\ \sin(\theta) & \cos(\theta) \end{pmatrix} \cdot \begin{pmatrix} \frac{1}{a} & 0 \\ 0 & \frac{1}{b} \end{pmatrix} \cdot \begin{pmatrix} \cos(\theta) & \sin(\theta) \\ -\sin(\theta) & \cos(\theta) \end{pmatrix}$$
$$\Leftrightarrow \Sigma^{-1} = \begin{pmatrix} \frac{b\cos^{2} \theta + a\sin^{2} \theta}{ab} & \frac{b\sin \theta\cos \theta - a\sin \theta\cos \theta}{ab} \\ \frac{b\sin \theta\cos \theta - a\sin \theta\cos \theta}{ab} & \frac{b\sin^{2} \theta + a\sin^{2} \theta}{ab}\end{pmatrix} $$

On injecte dans la formule de $f_Z(z)$ :
$$f_Z(z) = \frac{1}{2\pi \sqrt{ab}}exp(-\frac{1}{2}\begin{pmatrix}x-\mu_1 & y-\mu_2 \end{pmatrix}\cdot\begin{pmatrix} \frac{b\cos^{2} \theta + a\sin^{2} \theta}{ab} & \frac{b\sin \theta\cos \theta - a\sin \theta\cos \theta}{ab} \\ \frac{b\sin \theta\cos \theta - a\sin \theta\cos \theta}{ab} & \frac{b\sin^{2} \theta + a\sin^{2} \theta}{ab}\end{pmatrix}\begin{pmatrix}x-\mu_1 \\ y-\mu_2 \end{pmatrix})$$
$$ \Leftrightarrow f_Z(z) = \frac{1}{2\pi \sqrt{ab}}exp(-\frac{1}{2}\begin{pmatrix}x-\mu_1 & y-\mu_2 \end{pmatrix}\cdot\begin{pmatrix} \frac{b\cos^{2} \theta + a\sin^{2} \theta}{ab} & \frac{b\sin \theta\cos \theta - a\sin \theta\cos \theta}{ab} \\ \frac{b\sin \theta\cos \theta - a\sin \theta\cos \theta}{ab} & \frac{b\sin^{2} \theta + a\sin^{2} \theta}{ab}\end{pmatrix}\cdot\begin{pmatrix}x-\mu_1 \\ y-\mu_2 \end{pmatrix})$$
\begin{multline*} \Leftrightarrow f_Z(z) = \frac{1}{2\pi \sqrt{ab}}exp(-\frac{1}{2}\begin{pmatrix}x-\mu_1 & y-\mu_2 \end{pmatrix}\frac{1}{ab}\cdot\begin{pmatrix} b\cos^{2} \theta + a\sin^{2} \theta & b\sin \theta\cos \theta - a\sin \theta\cos \theta \\ b\sin \theta\cos \theta - a\sin \theta\cos \theta & b\sin^{2} \theta + a\sin^{2} \theta\end{pmatrix}\\\cdot\begin{pmatrix}x-\mu_1 \\ y-\mu_2 \end{pmatrix})\end{multline*}
\begin{multline*}\Leftrightarrow f_Z(z) = \frac{1}{2\pi \sqrt{ab}}exp(-\frac{1}{2ab}\begin{pmatrix}x-\mu_1 & y-\mu_2 \end{pmatrix}\cdot\begin{pmatrix} b\cos^{2} \theta + a\sin^{2} \theta & b\sin \theta\cos \theta - a\sin \theta\cos \theta \\ b\sin \theta\cos \theta - a\sin \theta\cos \theta & b\sin^{2} \theta + a\sin^{2} \theta\end{pmatrix}\\\cdot \begin{pmatrix}x-\mu_1 \\ y-\mu_2 \end{pmatrix})\end{multline*}

Calculons l'intérieur de l'exponentielle :

\[
    \Leftrightarrow \frac{{[(x - \mu_1) \cdot \cos(\theta) + (y - \mu_2) \cdot \sin(\theta)}]^2}{{2a \cdot \log(\frac{1}{2\pi K \sqrt{ab}})}} + \frac{{[(x - \mu_1) \cdot \sin(\theta)-(y - \mu_2) \cdot \cos(\theta)}]^2}{{2b \cdot \log(\frac{1}{2\pi K \sqrt{ab}})}} = 1
\]

\[
    \Leftrightarrow \frac{1}{2\pi \sqrt{ab}}exp(\frac{{[(x - \mu_1) \cdot \cos(\theta) + (y - \mu_2) \cdot \sin(\theta)}]^2}{{2a \cdot \log(\frac{1}{2\pi K \sqrt{ab}})}} + \frac{{[(x - \mu_1) \cdot \sin(\theta)-(y - \mu_2) \cdot \cos(\theta)}]^2}{{2b \cdot \log(\frac{1}{2\pi K \sqrt{ab}})}}) = 1
\]



Ici, le centre de l'ellipse est donné par $\mathbf{\mu} = (\mu_1, \mu_2)$, $\sqrt{2a \log(\frac{1}{2\pi K \sqrt{ab}})}$ est la demi-longueur de l'axe principal et $\sqrt{2b \log(\frac{1}{2\pi K \sqrt{ab}})}$ la demi-longueur de l'axe secondaire, $K$ est la constante de normalisation et $\theta$ est l'angle de rotation de l'ellipse.

\[
\Sigma = \begin{pmatrix}
    \sigma_{x}^2 & \sigma_{xy} \\
    \sigma_{xy} & \sigma_{y}^2 
\end{pmatrix}
\]

\end{document}

