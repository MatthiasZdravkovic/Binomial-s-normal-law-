\documentclass{article}
\usepackage{graphicx} % Required for inserting images
\usepackage{amsmath}
\usepackage{marginnote}
\usepackage{amssymb}

\title{Projet Maths}
\author{Matthias Zdravkovic, Ameen Mohd Fairuz}
\date{Mai 2023}

\begin{document}


\maketitle

\section{Développements mathématiques}

\reversemarginpar\marginnote{1. a)}[0cm]

On part de la loi normale bidimensionnelle :
$$f_Z(z) = \frac{1}{2\pi \sqrt{\det \Sigma}}exp(-\frac{1}{2}{}^{t}(z-\mu)\Sigma^{-1}(z-\mu))$$
On pose
$$\Sigma = U \cdot \Lambda \cdot U^T$$
$$ \Leftrightarrow \det \Sigma = \det U \cdot \det \Lambda \cdot \det (U^T) $$
\begin{center}
$(\det U)^{2}=1$ car c'est une matrice orthogonale
\end{center}
$$\Leftrightarrow \det \Sigma =\det \Lambda$$
avec $\Sigma \in M_2(\mathbb{R})$ matrice symétrique positive définie, $U \in M_2(\mathbb{R})$ matrice orthogonale et $\Lambda \in M_2(\mathbb{R})$ matrice diagonale à coefficients positifs.\\
On pose
\[
U = \begin{pmatrix}
    \cos(\theta) & -\sin(\theta) \\
    \sin(\theta) & \cos(\theta)
\end{pmatrix}
\]
et 

\[
\Lambda = \begin{pmatrix}
    a & 0 \\
    0 & b
\end{pmatrix}
\]
avec 
$$ \Sigma^{-1} = U \cdot \Lambda^{-1} \cdot U^{-1}$$
$$\Leftrightarrow \Sigma^{-1} = \begin{pmatrix} \cos(\theta) & -\sin(\theta) \\ \sin(\theta) & \cos(\theta) \end{pmatrix} \cdot \begin{pmatrix} \frac{1}{a} & 0 \\ 0 & \frac{1}{b} \end{pmatrix} \cdot \begin{pmatrix} \cos(\theta) & -\sin(\theta) \\ \sin(\theta) & \cos(\theta) \end{pmatrix}^{-1}$$
$$\Leftrightarrow \Sigma^{-1} = \begin{pmatrix} \cos(\theta) & -\sin(\theta) \\ \sin(\theta) & \cos(\theta) \end{pmatrix} \cdot \begin{pmatrix} \frac{1}{a} & 0 \\ 0 & \frac{1}{b} \end{pmatrix} \cdot \frac{1}{\cos^{2} \theta + \sin^{2} \theta}\begin{pmatrix} \cos(\theta) & \sin(\theta) \\ -\sin(\theta) & \cos(\theta) \end{pmatrix}$$
$$\Leftrightarrow \Sigma^{-1} = \begin{pmatrix} \cos(\theta) & -\sin(\theta) \\ \sin(\theta) & \cos(\theta) \end{pmatrix} \cdot \begin{pmatrix} \frac{1}{a} & 0 \\ 0 & \frac{1}{b} \end{pmatrix} \cdot \begin{pmatrix} \cos(\theta) & \sin(\theta) \\ -\sin(\theta) & \cos(\theta) \end{pmatrix}$$
$$\Leftrightarrow \Sigma^{-1} = \begin{pmatrix} \frac{b\cos^{2} \theta + a\sin^{2} \theta}{ab} & \frac{b\sin \theta\cos \theta - a\sin \theta\cos \theta}{ab} \\ \frac{b\sin \theta\cos \theta - a\sin \theta\cos \theta}{ab} & \frac{b\sin^{2} \theta + a\sin^{2} \theta}{ab}\end{pmatrix} $$

On injecte dans la formule de $f_Z(z)$ :
$$f_Z(z) = \frac{1}{2\pi \sqrt{ab}}exp(-\frac{1}{2}\begin{pmatrix}x-\mu_1 & y-\mu_2 \end{pmatrix}\cdot\begin{pmatrix} \frac{b\cos^{2} \theta + a\sin^{2} \theta}{ab} & \frac{b\sin \theta\cos \theta - a\sin \theta\cos \theta}{ab} \\ \frac{b\sin \theta\cos \theta - a\sin \theta\cos \theta}{ab} & \frac{b\sin^{2} \theta + a\sin^{2} \theta}{ab}\end{pmatrix}\begin{pmatrix}x-\mu_1 \\ y-\mu_2 \end{pmatrix})$$
\begin{multline*}\Leftrightarrow f_Z(z) = \frac{1}{2\pi \sqrt{ab}}exp(-\frac{1}{2ab}\begin{pmatrix}x-\mu_1 & y-\mu_2 \end{pmatrix}\cdot\begin{pmatrix} b\cos^{2} \theta + a\sin^{2} \theta & b\sin \theta\cos \theta - a\sin \theta\cos \theta \\ b\sin \theta\cos \theta - a\sin \theta\cos \theta & b\sin^{2} \theta + a\sin^{2} \theta\end{pmatrix}\\\cdot \begin{pmatrix}x-\mu_1 \\ y-\mu_2 \end{pmatrix})\end{multline*}

Calculons le produit matriciel A dans l'exponentielle :

$$ A=\begin{pmatrix}x-\mu_1 & y-\mu_2 \end{pmatrix}\cdot\begin{pmatrix} b\cos^{2} \theta + a\sin^{2} \theta & b\sin \theta\cos \theta - a\sin \theta\cos \theta \\ b\sin \theta\cos \theta - a\sin \theta\cos \theta & b\sin^{2} \theta + a\sin^{2} \theta\end{pmatrix}\cdot \begin{pmatrix}x-\mu_1 \\ y-\mu_2 \end{pmatrix}$$ 
\begin{multline*} \Leftrightarrow A=((a\sin^2\theta+b\cos^2\theta)(x-\mu_1)+\cos\theta\sin\theta(b-a)(y-\mu_2))(x-\mu_1)\\+((b\sin^2\theta+a\cos^2\theta)(y-\mu_2)+\cos\theta\sin\theta(b-a)(x-\mu_1))(y-\mu_2)\end{multline*}
\begin{multline*} \Leftrightarrow A=(a\sin^2\theta+b\cos^2\theta)(x-\mu_1)^2+\cos\theta\sin\theta(b-a)(x-\mu_1)(y-\mu_2)\\+(b\sin^2\theta+a\cos^2\theta)(y-\mu_2)^2+\cos\theta\sin\theta(b-a)(x-\mu_1)(y-\mu_2)\end{multline*}
\begin{multline*} \Leftrightarrow A=(a\sin^2\theta+b\cos^2\theta)(x-\mu_1)^2+2\cos\theta\sin\theta(b-a)(x-\mu_1)(y-\mu_2)\\+(b\sin^2\theta+a\cos^2\theta)(y-\mu_2)^2\end{multline*}
\begin{multline*} \Leftrightarrow A=a\sin^2\theta(x-\mu_1)^2+b\cos^2\theta(x-\mu_1)^2+2\cos\theta\sin\theta(b-a)(x-\mu_1)(y-\mu_2)\\+b\sin^2\theta(y-\mu_2)^2+a\cos^2\theta(y-\mu_2)^2\end{multline*}
\begin{multline*} \Leftrightarrow A=a(\sin^2\theta(x-\mu_1)^2+\cos^2\theta(y-\mu_2)^2)+2\cos\theta\sin\theta(b-a)(x-\mu_1)(y-\mu_2)\\+b(\sin^2\theta(y-\mu_2)^2+\cos^2\theta(x-\mu_1)^2)\end{multline*}
\begin{multline*} \Leftrightarrow A=a((x-\mu_1)^2\sin^2\theta+(y-\mu_2)^2\cos^2\theta-2\cos\theta\sin\theta(x-\mu_1)(y-\mu_2))\\+b((y-\mu_2)^2\sin^2\theta+(x-\mu_1)^2\cos^2\theta+2\cos\theta\sin\theta(x-\mu_1)(y-\mu_2))\end{multline*}
$$ \Leftrightarrow A=a((x-\mu_1)\sin\theta-(y-\mu_2)\cos\theta)^{2}+b((x-\mu_1)\cos\theta+(y-\mu_2)\sin\theta)^{2}$$

On obtient donc :

$$ f_Z(z) = \frac{1}{2\pi \sqrt{ab}}exp(-\frac{((x-\mu_1)\sin\theta-(y-\mu_2)\cos\theta)^{2}}{2b}-\frac{((x-\mu_1)\cos\theta+(y-\mu_2)\sin\theta)^{2}}{2a})=K$$

$$\Leftrightarrow   \frac{((x-\mu_1)\sin\theta-(y-\mu_2)\cos\theta)^{2}}{2b}+\frac{((x-\mu_1)\cos\theta+(y-\mu_2)\sin\theta)^{2}}{2a}=-\ln(K2\pi\sqrt{ab})$$

$$\Leftrightarrow   \frac{((x-\mu_1)\sin\theta-(y-\mu_2)\cos\theta)^{2}}{2b\ln(\frac{1}{K2\pi\sqrt{ab}})}+\frac{((x-\mu_1)\cos\theta+(y-\mu_2)\sin\theta)^{2}}{2a\ln(\frac{1}{K2\pi\sqrt{ab}})}=1$$

Ici, le centre de l'ellipse est donné par $\mathbf{\mu} = (\mu_1, \mu_2)$, $\sqrt{2a \log(\frac{1}{2\pi K \sqrt{ab}})}$ est la demi-longueur de l'axe principal et $\sqrt{2b \log(\frac{1}{2\pi K \sqrt{ab}})}$ la demi-longueur de l'axe secondaire, $K$ est la constante de normalisation et $\theta$ est l'angle de rotation de l'ellipse.

\reversemarginpar\marginnote{1. b)}[0cm]

Pour calculer la probabilité qu'un point tiré selon la loi $Z$ appartienne à la surface interne $S_k$ délimitée par l'ellipse d'isodensité $K$, on intègre la densité de $Z$ par la surface de l'ellipse:
$$P(Z \in S_k) = \int \frac{1}{\sqrt{2\pi^2det\Sigma}}exp(-\frac{1}{2} {}^{t}(z-\mu)\Sigma^{-1}(z-\mu)) dz$$
$$\Leftrightarrow \int\int_{S_k}^{} \frac{1}{2\pi\sqrt{ab}}exp(-\frac{[(x-\mu_1)cos\theta+(y-\mu_2)sin\theta]^2}{2a}-\frac{[(x-\mu_1)sin\theta-(y-\mu_2)cos\theta]^2}{2b})dxdy$$

Pour simplifier les calculs, on fait un changement de variable:

$$\left\{\begin{matrix}x_1=\frac{(x-\mu_1)cos\theta+(y-\mu_2)sin\theta}{\sqrt{2a}}
    \\ x_2=\frac{(x-\mu_1)sin\theta-(y-\mu_2)cos\theta}{\sqrt{2a}}
    \end{matrix}\right.$$
$$\Leftrightarrow 
\left\{\begin{matrix}x=\sqrt{2a}cos\theta x_1+\sqrt{2b}sin\theta y_1 +\mu_1
\\ y=\sqrt{2a}sin\theta x_1-\sqrt{2b}cos\theta y_1 +\mu_2
\end{matrix}\right.$$
$$\Leftrightarrow 
\frac{\partial x}{\partial x_1}=\sqrt{2a}cos\theta\; ;\; \frac{\partial x}{\partial y_1}=\sqrt{2b}sin\theta$$
$$
\frac{\partial y}{\partial x_1}=\sqrt{2a}sin\theta\; ;\; \frac{\partial y}{\partial y_1}=-\sqrt{2b}cos\theta$$

En changeant $(x,y)$ à $(x_1,y_1)$, nous avons maintenant:

$$\int\int_{S_k}^{} \frac{1}{2\pi\sqrt{ab}}exp(-x_1^2-y_1^2)\left| \begin{matrix}\frac{\partial x}{\partial x_1}
    &  \frac{\partial x}{\partial y_1}
    &  \\\frac{\partial y}{\partial x_1}
    & \frac{\partial y}{\partial y_1}
\end{matrix} \right|dx_1dy_1$$
$$\Leftrightarrow \int\int_{S_k}^{} \frac{1}{2\pi\sqrt{ab}}exp(-x_1^2-y_1^2)\left| -\sqrt{4ab}cos^2\theta -\sqrt{4ab}sin^2\theta\right|dx_1dy_1$$
$$\Leftrightarrow \int\int_{S_k}^{} \frac{2\sqrt{ab}}{2\pi\sqrt{ab}}exp(-x_1^2-y_1^2)dx_1dy_1$$
$$\Leftrightarrow \int\int_{S_k}^{} \frac{1}{\pi}exp(-x_1^2-y_1^2)dx_1dy_1$$

Changement en base polaire: on utilise $x_1=rcos\varphi$ et $y_1=rsin\varphi$

$$\int\int_{S_k}^{} \frac{1}{\pi}exp(-r^2)rdrd\varphi$$
$$\Leftrightarrow \frac{1}{\pi}\int_{0}^{r}\int_{0}^{2\pi} re^{-r}d\varphi dr$$
$$\Leftrightarrow \int_{0}^{r} 2re^{-r} dr$$

Or, ici il faut utiliser $r=$ l'équation de l'ellipse en base $(r,\varphi)$

Pour cela, nous reprenons l'équation de l'ellipse trouvée à la question 1.a)
$$\frac{((x-\mu_1)\sin\theta-(y-\mu_2)\cos\theta)^{2}}{2b\ln(\frac{1}{K2\pi\sqrt{ab}})}+\frac{((x-\mu_1)\cos\theta+(y-\mu_2)\sin\theta)^{2}}{2a\ln(\frac{1}{K2\pi\sqrt{ab}})}=1$$
$$\Leftrightarrow \frac{x_1^2}{\ln(\frac{1}{K2\pi\sqrt{ab}})}+\frac{y_1^2}{\ln(\frac{1}{K2\pi\sqrt{ab}})}=1$$
$$\Leftrightarrow \frac{r^2}{\ln(\frac{1}{K2\pi\sqrt{ab}})}=1$$
$$\Leftrightarrow r=\sqrt{\ln(\frac{1}{K2\pi\sqrt{ab}})}$$

Nous utilisons cette équation de $r$ comme borne superieure de notre intégrale:
$$\int_{0}^{\sqrt{\ln(\frac{1}{K2\pi\sqrt{ab}})}} 2re^{-r} dr$$
$$\Leftrightarrow \left[ -e^{r^2} \right]_{0}^{\sqrt{\ln(\frac{1}{K2\pi\sqrt{ab}})}}$$
$$\Leftrightarrow -e^{\sqrt{\ln(\frac{1}{K2\pi\sqrt{ab}})}}+e^0$$

Nous trouvons donc:

$$p=P(Z \in S_k)$$
$$\Leftrightarrow p=1-2\pi K\sqrt{ab}$$





\reversemarginpar\marginnote{2.}[0cm]
\[
\Sigma = \begin{pmatrix}
    \sigma_{x}^2 & \sigma_{xy} \\
    \sigma_{xy} & \sigma_{y}^2 
\end{pmatrix}
\]

\end{document}

